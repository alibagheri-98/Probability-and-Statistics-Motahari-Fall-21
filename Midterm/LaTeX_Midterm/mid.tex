
\documentclass[a4paper,11pt]{article}

%\usepackage{fullpage,amsmath,clrscode,amsfonts,graphicx}
%\usepackage{mathtools}
%\usepackage{tabularx}
%\usepackage{adjustbox}
%\usepackage{setspace}

\usepackage{fullpage,amsmath,hyperref,color,clrscode,amsfonts,graphicx,textcomp}
\usepackage{tabularx}
\usepackage{adjustbox}
\usepackage{setspace}
\usepackage{upgreek}
\usepackage{amsmath}
\usepackage{amssymb}
\usepackage{tikz}
\usepackage{amsmath}


\usepackage{comment}
\usepackage[stable]{footmisc}



\includecomment{Answer}
%\excludecomment{Answer}

\usepackage[a4paper, total={6.8in, 8.7in}]{geometry}
\addtolength{\textwidth}{0.2in}
\addtolength{\oddsidemargin}{-0.1in}
\addtolength{\evensidemargin}{-0.1in}

\addtolength{\textheight}{0.5in}
\addtolength{\topmargin}{-0.25in}
\linespread{1.5}

\makeatletter
  \def\hrulefill#1{\leavevmode\leaders\hrule\@height#1\hfill \kern\z@}
\makeatother
%\defpersianfont\iranic[Scale=1.1]{XB Zar Oblique} 	% define Iranic font

\usepackage{float}
\usepackage{xepersian}
\usepackage{fontspec}
\newfontfamily\DejaSans{DejaVu Sans}
\usepackage{graphicx}
\usepackage{array}
\newcolumntype{P}[1]{>{\centering\arraybackslash}p{#1}}





% -------------------- Fonts --------------------

\settextfont[
Scale=1.2,
Extension=.ttf, 
Path=styles/fonts/,
BoldFont=XB NiloofarBd,
ItalicFont=XB NiloofarIt,
BoldItalicFont=XB NiloofarBdIt
]{XB Niloofar}

{%
    \ExplSyntaxOn
    \cs_set_eq:NN
    \etex_iffontchar:D
    \tex_iffontchar:D
    \ExplSyntaxOff

\setmathdigitfont [
Scale=1.2,
Extension=.ttf, 
Path=styles/fonts/,
BoldFont=XB NiloofarBd,
ItalicFont=XB NiloofarIt,
BoldItalicFont=XB NiloofarBdIt
]{XB Niloofar} %
}

\defpersianfont\sayeh[
Scale=1,
Path=styles/fonts/
]{XB Kayhan Pook}


% -------------------- end Fonts --------------------

% % % % % % % % % % % % % % % % % % % % % % % % % % % % % % % % % % % % % % % % % %

\begin{document}

%--------------------------------------------------------------
\centering{\footnotesize{بسمه تعالی}}
\newcolumntype{R}{>{\raggedright\arraybackslash}X}%
\newcolumntype{L}{>{\raggedleft\arraybackslash}X}%
\begin{tabularx}{\textwidth}{LR}
\hspace{3mm} 
%\adjustbox{valign=m}{\includegraphics[scale=0.5]{logo.pdf}}
\raisebox{-.5\height}{\includegraphics[scale=0.5]{logo.pdf}} \vspace{-3mm}& {\Large \textbf{آمار و احتمال مهندسی}} \\
 	& {\textbf{نیم‌سال اول ۱۴۰۰-۱۴۰۱}} \\
\footnotesize{\textbf{دانشکده مهندسی کامپیوتر}} 										& {
مدرس:  سید ابوالفضل مطهری
 }\\
\end{tabularx}
%--------------------------------------------------------------

\vspace{5mm}

%--------------------------------------------------------------
\rule{\textwidth}{1.4pt}
\\
\vspace{-1mm}
%\today
\hfill
{\text{امتحان میان‌ترم(- نمره)}}
\hfill
\vspace{-1mm}
\\
\rule{\textwidth}{1.4pt}
%--------------------------------------------------------------

\flushright
\indent
%=========================== Problems =========================
%\vspace*{-1cm}

\section*{ 
\textbf {سوال1: نمرات امتحان (- نمره)  }
\\}

استاد درسی ، دانشجویان کلاس را با توجه به نمرات تمارین‌شان به دو دسته‌ی متوسط و زرنگ تقسیم کرده است که به ترتیب 70 و 30 درصد کل دانشجویان هستند. احتمال اینکه دانشجویان متوسط و زرنگ در امتحان زیر 10 شوند به ترتیب 3.0 و 1.0 است. یکی از دانشجویان کلاس را به صورت تصادفی انتخاب می‎کنیم. احتمال آن که این دانشجو در امتحان پایان‎ترم زیر 10 شود به شرط آنکه در امتحان میان‎ترم نیز زیر 10 شده باشد را بیابید. 
واقعه زیر 10 شدن در میان‎ترم و پایان‎ترم به شرط زرنگ بودن (ونیز به شرط متوسط بودن) را مستقل از هم در نظر بگیرید.


\vspace{2mm}
\hrule
\vspace{8mm}
			  						  			  
%%%%%%%%%%%%%%%%%%%%%%%%%%%%%%%%%%%%%%%%%%%%%%%%%%%%%%%%
\section*{ 		
\textbf {سوال دوم: تابع چکیده ساز (- نمره)  }
\\}

تابع چکیده ساز \lr{(hash function)} یکی از توابعی است که در رمز نگاری از آن استفاده می‌شود. ورودی این تابع یک پیام به طول دلخواه است و خروجی آن نیز، یک پیام با طول ثابت می‌باشد که به آن hash یا اثر انگشت دیجیتال نیز گفته می‌شود. چون طول ورودی دلخواه است ولی طول خروجی ثابت است بنابراین تابعی یک به یک نیست و ممکن است دو پیام اثر انگشت دیجیتال یکسانی داشته باشند. در چنین حالتی می‌گویند "برخورد" رخ داده است. حمله جستجوی جامع یکی از حملاتی است که به توابع چکیده ساز صورت می‌گیرد. در این حمله اثر انگشت دیجیتال تعداد زیادی پیام محاسبه می‌شود تا برخورد رخ دهد. یعنی دو پیام متفاوت با اثر انگشت یکسان پیدا شود. اگر طول خروجی برابر با N بیت باشد، تعداد حداقل پیام‌هایی که باید اثر انگشت دیجیتال آن‌ها حساب شود تا با احتمال حداقل $\frac{1}{2}$ دو پیام با اثر انگشت یکسان یافت شود را بیابید. 


\vspace{2mm}
\hrule
\vspace{4mm}

%%%%%%%%%%%%%%%%%%%%%%%%%%%%%%%%%%%%%%%%%%%%%%%%%%%%%%%%
\newpage 
\section*{
\textbf { سوال سوم: تاس هرمی (- نمره)  }
\\ }

در کیسه بزرگی سه نوع تاس هرمی شکل \lr{ A,B,C } وجود دارد  که ظاهرا همگی مثل هم هستند. وجوه تاس‌ها با اعداد 1و2و3و4 شماره‌گذاری شده است و احتمال ظاهر شدن هر یک از وجه‌های مختلف در هر یک از این تاس‌ها مطابق جدول زیر است:
  \begin{table}[H]
  \centering
  \caption{مشخصات تاس‌ها}
  \begin{tabular}{|P{.2\textwidth}|P{.2\textwidth}|P{.2\textwidth}|P{.2\textwidth}|}\hline
   \textbf{وجه/نوع} & \textbf{A} & \textbf{B}       & \textbf{C} \\ \hline
    1     & 2.0             &1.0              & 25.0             \\ \hline
    2     & 3.0             & 7.0              & 3.0             \\ \hline
    3     & 4.0            & 05.0              &05.0             \\ \hline
    4     & 1.0            & 15.0           &4.0             \\ \hline

  \end{tabular}
  \end{table}
تاس های A و B و C به ترتیب 30 و 50 و 20 درصد کل تاس ها را تشکیل می‌دهند. بصورت کاملا تصادفی تاسی را از کیسه بر می‌داریم و سپس تاس را می‌اندازیم. بر اساس نتیجه انداختن تاس می‌خواهیم با کمترین احتمال خطا تصمیم بگیریم که تاس از کدام یک از این سه نوع است. بر اساس نتیجه انداختن تاس تصمیم شما در مورد نوع تاس چه خواهد بود و احتمال خطای استراتژی شما چقدر است؟ 


%%%%%%%%%%%%%%%%%%%%%%%%%%%%%%%%%%%%%%%%%%%%%%%%%%%%%%%

%%%%%%%%%%%%%%%%%%%%%%%%%%%%%%%%%%%%%%%%%%%%%%%%%%%%%%%
~\\[4mm]
\hrule
\hfill موفق باشید.
%%%%%%%%%%%%%% END OF PROBLEMS %%%%%%%%%%%%%%%

\end{document}
